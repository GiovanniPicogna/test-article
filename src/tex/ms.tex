\documentclass{aa}

\usepackage{showyourwork}
\usepackage{hyperref}
\hypersetup{
  colorlinks,
  citecolor=cyan,
  linkcolor=red,
  urlcolor=blue
}
\usepackage{graphicx}
\usepackage{txfonts}
\usepackage{orcidlink}
\usepackage{siunitx}
% common astronomical symbols
\let\sun\odot
\DeclareSIUnit\erg{erg}
\DeclareSIUnit\jansky{Jy}
\DeclareSIUnit\year{yr}
\DeclareSIUnit\solarluminosity{\ensuremath{L_\sun}}
\DeclareSIUnit\solarmass{\ensuremath{M_\sun}}
\DeclareSIUnit\solarradius{\ensuremath{R_\sun}}
\DeclareSIUnit\arcsecond{as}
\DeclareSIUnit\astronomicalunit{au}

\defcitealias{2019MNRAS.487..691P}{Paper~I}
\defcitealias{2021MNRAS.508.1675E}{Paper~II}
\defcitealias{2021MNRAS.508.3611P}{Paper~III}

\begin{document}

   \title{Disks in transition}

   \subtitle{I. Internal photoevaporation and disk dispersal}

   \author{Giovanni Picogna \orcidlink{0000-0003-3754-1639}}

   \institute{Universitäts-Sternwarte, Ludwig-Maximilians-Universität München, Scheinerstr. 1, D-81679 München, Germany\\
   \email{picogna@usm.lmu.de}
    }

   \date{Received \today; accepted \today}

   \abstract
      {The strong X-ray irradiation from young solar-type stars may play a crucial role in the thermodynamics and chemistry of circumstellar discs, driving their evolution in the last stages of disc dispersal when the accretion rates onto the central star drop and thermal winds become the only viable way to dissipate the disc. Yet, we have discovered a large population of disks with cavities that are not in transition, and no concrete evidence of a disc in his last evolutionary stages.}
      {We study the influence of the cavity sizes on the disc mass-loss rates due to X-ray irradiation for stars with masses between $0.1$ and \SI{1}{\solarmass} and derive observables to be compared with the transition discs population.}
      {We ran a series of radiative-hydrodynamical simulations to derive the equilibrium wind mass-loss rate as a function of cylindrical radius from the central star for varying stellar and disc properties, and apply the resulting profiles to 1D long term disc evolution. We compared then the synthetic transition disc population with the observed one.}
      {We found a simple scaling between cavity size, stellar mass and disc mass that can be readily applied to 1D viscous evolution codes to model the final disc dispersal in a proper fashion.}
      {We provide new analytical relations for the mass-loss rates and profiles of photoevaporative winds as a function of the stellar mass that can be used in disc and planet population synthesis models. Our photoevaporative models correctly predict the observed trend of inner-disc lifetime as a function of stellar mass with an increased steepness for stars smaller than \SI{0.3}{\solarmass}, indicating that X-ray photoevaporation is a good candidate to explain the observed disc dispersal process.}

   \keywords{}

   \maketitle

   \section{Introduction} \label{sec:intro}

   Our understanding of the physical processes driving the evolution of protoplanetary discs and ultimately the formation of planets is based on the observational evidence of disc lifetimes.
\citet{1989AJ.....97.1451S} found that accretion discs are ubiquitous around young stellar objects (\SI{\sim 1}{Myr}). The initial disc fraction is almost independent of stellar mass (from $0.1$ up to \SI{10}{\solarmass}) and stellar environment (\citet{2000AJ....120.3162L}, \citet{2006A&A...451..177B}).
After \SI{10}{Myr} the picture changes completely since more than \SI{90}{\percent} of stars show no emission within \SI{1}{\astronomicalunit} \citep{2004ApJ...612..496M}, and emission from small grains is found only in a few per cent of discs (\citet{2004ApJ...608..526L}, \citet{2005AJ....129.1049C}, \citet{2006ApJ...639.1138S}).
This sets an upper limit to the disc dispersal process within $10$ to \SI{20}{Myr} (\citet{2007ApJ...671.1784H}, \citet{2010A&A...510A..72F}, \citet{2014A&A...561A..54R}).
There are two main indirect evidences of inside-out disc dispersal. \citet{2013MNRAS.428.3327K} found that $\sim30\%$ of discs in nearby star forming regions evolve in the colour-colour plane consistently with an inside-out disc dispersal. Moreover, fitting the fraction of discs with dust emission at different wavelengths, \citet{2014A&A...561A..54R} found an e-folding time of \SIrange{2}{3}{Myr} at \SIrange[]{3}{12}{\micro\meter} and \SIrange[]{4}{6}{\mega\year} at \SIrange[]{22}{24}{\micro\meter}, hinting at an inside-out dispersal or to a second generation dust production.
The dust disc lifetime as a function of the stellar mass is less well characterised but the dispersal timescale appears to be up to two times faster for high mass stars ($\geq$ \SI{2}{\solarmass}, \cite{2015A&A...576A..52R}).
The mass accretion onto the central star has also been studied extensively, though its evolution is less well constrained. Nevertheless, it has been shown that the characteristic timescale of disc accretion is shorter than that of dust disc dispersal (\cite{2006ApJ...648.1206J}, \cite{2010A&A...510A..72F}), and the accretion rate falls off as a function of time with a power law (e.g., \cite{2012ApJ...755..154M}, \cite{2014A&A...572A..62A}, \cite{2016ARA&A..54..135H}).

The main physical process driving disc dispersal is still largely unconstrained, and it might change during the disc evolution and at different locations in the disc. A large body of evidence is pointing at magnetic disc winds as the main mechanism responsible for angular momentum transfer and mass-loss \citep{2016ApJ...821...80B}, particularly at early times. Photoevaporative disc winds may also co-exist at early times and perhaps dominate at later stages (\cite{2017RSOS....470114E}; \cite{2020MNRAS.496..223W}). From a theoretical standpoint, there is a push to obtain better models that can be linked to observables to test their validity.

In the first paper of this series (\cite{2019MNRAS.487..691P}, hereafter \citetalias{2019MNRAS.487..691P}) we showed the dependence between the stellar X-ray luminosity and the mass-loss rate due to thermal winds generated by the XEUV heating from the central star. Then we studied the influence of carbon depletion \citep{2019MNRAS.490.5596W} and stellar spectra hardness on the mass-loss rates (\cite{2021MNRAS.508.1675E}, hereafter \citetalias{2021MNRAS.508.1675E}). Finally we looked in the dependance of mass-loss rate and stellar mass (\cite{2021MNRAS.508.3611P}, hereafter \citetalias{2021MNRAS.508.3611P}). In this following work, we bring all the previous results together and study the influence of the gap hole radii dependence and the implications to interpret current and future observations of transition discs.

The most important direct observational evidence we have in support of a disk-in-transition phase are the observations of disks with cavities and low accretion rates, and the population of transition discs that show forbidden emission lines just in the low velocity component \citep{2020ApJ...903...78P}.

If there is a disk phase when internal photoevaporation might not be hindered by the large column density of an inner magnetohydrodynamic (MHD) wind, this is the final clearing stage. It is then of paramount importance to study this critical, although short-lived, phase in the disk evolution.
The sample of disk in transition is not to be confused with the gapped disks population that are much more common and include a large fraction of highly accreting system where the cavity is most likely carved by forming giant planets that can effectively halt the dust but not the gas component.

In Section~\ref{sec:methods} we briefly discuss the numerical set-up adopted, following our previous work. We then describe the main results in Section~\ref{sec:results} and discuss their theoretical and observational implications in Section~\ref{sec:discussion}. The main conclusions are then drawn in Section~\ref{sec:conclusions}.

   \section{Dependence on the cavity size}\label{sec:cavity}

   \section{Methods}\label{sec:methods}

      \subsection{Radiative transfer}

      \subsection{Hydrodynamical model}\label{sec:hydro-model}

      \subsection{Viscous evolution}

   \section{Results}\label{sec:results}

      \subsection{Disc profiles}

      \subsection{Mass-loss rates}\label{sec:mdot}

      \subsection{Wind profiles}\label{sec:wind-prof}

      \subsection{Gas cavity radius dependence}\label{sec:gap-dependance}

   \section{Discussion}\label{sec:discussion}

      \subsection{Comparison with observations}

      \subsection{Comparison with other models}

   \section{Conclusions}\label{sec:conclusions}

   \section{Data availability}
      The data underlying this article and the scripts used to create the Figures are available at \ldots

   \begin{appendix}

   \end{appendix}

   \begin{acknowledgements}
      GP acknowledges support from the DFG Research Unit 'Transition Disks' (FOR 2634/2).
      This work was performed on the computing facilities of the Computational Center for Particle and Astrophysics (C2PAP).
      GP would like to thank Kristina Monsch and Tommaso Grassi for their insightful comments on the manuscript and during the development of the project.
      This research was supported by the Excellence Cluster ORIGINS which is funded by the Deutsche Forschungsgemeinschaft (DFG,  German Research Foundation) under Germany's Excellence Strategy - EXC-2094 - 390783311.
      This paper utilizes the D'Alessio Irradiated Accretion Disk (DIAD) code. We wish to recognize the work of Paola D'Alessio, who passed away in 2013. Her legacy and pioneering work live on through her substantial contributions to the field.
   \end{acknowledgements}

   \bibliographystyle{aa} % style aa.bst
   \bibliography{bib}

\end{document}
